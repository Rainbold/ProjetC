\documentclass[11pt]{article}

\usepackage[left=0.75in, right=0.75in, top=0.75in, bottom=0.75in]{geometry}
\usepackage{layout}
\usepackage{graphicx}
\usepackage[utf8]{inputenc}


\title{\textbf{PG110 Project}\\Bomberman}
\author{Maxime PETERLIN - Gabriel VERMEULEN \\\\{ENSEIRB-MATMECA, Bordeaux}}
\date{June, 6th 2014}


\begin{document}

\maketitle

\section{Fonctionnalités de bases}

RAS

\section{Fonctionnalités ajoutées}
	\subsection{Menus}
		Ajout de menus permettant de choisir, par exemple, entre le singleplayer et le multiplayer ou encore entre lancer une nouvelle partie ou en charger une.
	\subsection{Pathfinding pour les monstres}
		Implémentation de l'algorithme de Dijkstra. Le monstre suit le joueur en fonction de son agressivité qui définie ici comme étant le nombre de cases séparant le monstre et le joueur.
	\subsection{Animations}
		Implémentation de la gestion de l'animation des sprites.
	\subsection{Multijoueurs LAN (possibilité d'utiliser des Wiimotes)}
		Création d'un mode multijoueurs (maximum 4 joueurs) se jouant en 3 manches gagnantes.\\
		Il est également possible (et conseillé) d'utiliser des Wiimotes pour contrôler les déplacements des joueurs.
	\subsection{Changement des sprites et ajout de nouveaux bonus}
		Les sprites ont été modifiés par rapport à ceux proposés initialement.\\
		Les bonus qui ont été ajoutés sont les suivants :
		\begin{itemize}
			\item boost de vitesse.
		\end{itemize}
	\subsection{Editeur de maps}
		Création d'un éditeur de maps à l'aide de la bibliothèque GTK+.\\
		Prend uniquement en compte le format des maps que nous avons mis en place.

\end{document}
